\documentclass[varwidth=29cm,border=7pt]{standalone}
\usepackage{planning}

% =================================================================
% les paramètres du planning et les styles des cours, td, ...
% =================================================================
\tikzset{
  largeur du planning = 21cm,
  hauteur du planning = 17cm,
  largeur du nom du jour = 4cm,
  largeur du nom du groupe = 1cm,
  hauteur du deplacement vertical des heures = 4mm,
  nombre de jours = 5,
  nombre d'heures=11, % nombre d'heures par jour
  heure du debut=8,
  % -----------------------------------------------------
  % il y a aussi les style optionnels
  % -----------------------------------------------------
  % etiquette jour
  % etiquette groupe
  % etiquette heure
  % etiquette demi-heure
  % ligne jour
  % ligne groupe
  % ligne heure
  % ligne demi-heure
  % lignes verticales
  % -----------------------------------------------------
  cours/.style={
    titre/.append style={echelle=1.05,font=\bfseries},
    sur-titre/.append style={echelle=.85},
    sous-titre/.append style={echelle=.7},
    line width=.7pt,
  },
}

% =================================================================
% les commandes basées sur \creneau
%   \creneau[style]{surtitre}{titre}{sous-titre}{jour}{groupe}{horaires}
% =================================================================
\newcommand{\cours}[5][]{% \cours{groupe}{jour}{08:00-10:00}{sous-titre}
  \creneau[cours,#1]{\titremodule}{Cours \numeromodule}{#5}{#3}{#2}{#4}%
}
\newcommand{\autre}[5][]{% \autre{groupe}{jour}{08:00-10:00}{sous-titre}
  \creneau[cours,#1]{\titremodule}{\numeromodule}{#5}{#3}{#2}{#4}%
}

\begin{document}
  \begin{center}
    \Huge CAPES - Mathématiques - 2020/21 « Planning approximatif » du S1
  \end{center}

  \begin{planning}[cacher etiquettes groupes,screen colors 1]

    % ---------------------------------------------------------------
    \begin{module}{UE 5}{C.~Caux}
      \cours{}{lundi}{09:00-12:00}{}
    \end{module}
    % ---------------------------------------------------------------
    \begin{module}{UE 2}{G.~Jouves}
      \cours{}{lundi}{13:30-16:30}{}
    \end{module}
    % ---------------------------------------------------------------
    \begin{module}{UE 1}{Math pour l'enseignement}
      \cours{}{mardi}{08:00-12:00}{}
      \cours{1/2}{mercredi}{13:30-16:30}{}
      \cours{1/2}{jeudi}{13:30-16:30}{}
    \end{module}
    % ---------------------------------------------------------------
    \begin{module}{Anglais}{N.~Chapel}
      \cours{}{mardi}{13:30-16:30}{SUP - salles 11/12}
    \end{module}
    \begin{module}{Leçons}{}
    % ---------------------------------------------------------------
      \autre{}{mardi}{16:30-18:30}{}
      \autre{2/2}{mercredi}{09:00-12:00}{si pas de problème}
      \autre{2/2}{mercredi}{13:30-16:30}{si pas de cours}
      \autre{2/2}{jeudi}{13:30-16:30}{si pas de cours}
      \autre{2/2}{vendredi}{13:30-16:30}{si pas de cours}
    \end{module}
    % ---------------------------------------------------------------
    \begin{module}{Problème}{}
      \autre{1/2}{mercredi}{08:00-13:00}{5 heures}
    \end{module}
    % ---------------------------------------------------------------
    \begin{module}{TICE}{T.~Loof}
      \autre{}{jeudi}{08:00-10:00}{Groupe 1}
      \autre{}{jeudi}{10:00-12:00}{Groupe 2}
    \end{module}
    % ---------------------------------------------------------------
    \begin{module}{UE4}{ESPE de Villeneuve d'Ascq}
      \autre{}{vendredi}{08:30-11:30}{}
    \end{module}
    % ---------------------------------------------------------------
    \begin{module}{UE3}{Recherche}
      \autre{1/2}{vendredi}{13:30-16:30}{}
    \end{module}
    % ---------------------------------------------------------------
  \end{planning}

  \begin{itemize}
    \item La plupart des créneaux sont susceptibles de varier d'une semaine à l'autre.
    \item Pour connaître le « vrai » planning de la semaine il faut se rendre dans le document Google Docs.
  \end{itemize}
\end{document}
